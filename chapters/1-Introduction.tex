\section{Introduction}
Weak memory specializes on research that aims to improve the results in memory modeling. Weak memory focuses on determining the relative strength of different models. This is why one of the common challenges is to show equaivalences between several models. Given two memory models \texttt{A} and \texttt{B} we might want to check if model \texttt{A} is stronger then model \texttt{B}. If that condition holds, then the consistency of the model \texttt{A} would imply the consistency of the model \texttt{B}.

Memory models and propositions about them are usually represented as expressions over relational language. However, proofs of such the propositions are typically massive and very error prone. There were several cases of incorrect result in submited and published papers. Batty et al.~\cite{batty_2016}\ suggested an incorrect fix for the semantics of SC calls in C++, which was later documented by Lahav et al.~\cite{lahav2017repairing}. Moreover Pichon-Pharabod et al.\  in 2016 suggested an incorrect proof of compilation in their paper~\cite{PichonPharabod_Sewell16}. 

Weak memory proofs are usually written in Coq~\cite{bertot2013interactive}, which is a proof assistant system. Coq helps to grant the correctness of the proof process. Coq provides a language called Calculus of Inductive Constructions (CIC), which is used to write the proofs. In CIC, the proofs are expressed as formal mathematical objects, and the correctness of the proof is ensured by checking that the proof is consistent with the axioms and rules of logic. Coq is the standard of development in the field of weak memory, helping to unify and verify already complex proofs. 

Nevertheless, weak memory proofs, even with a use of Coq, tend to be huge and convoluted. This is why in this thesis, we are focusing on automating a specific part of weak memory proofs, namely the proofs of equaivalences between several memory models. As already mentioned, memory models are defined as relations. Let's denote \texttt{r} and \texttt{r'} the two memory models or relations over a given set. Now we may consider common operations in relational language, e.g.\ transitive closure (\texttt{r*}), reflexive closure (\texttt{r?}), or composition (\texttt{r ;; r'}). 